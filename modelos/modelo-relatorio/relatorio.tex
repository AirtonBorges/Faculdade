\documentclass[a4paper, 12pt]{article}

\usepackage[portuges]{babel}
\usepackage[utf8]{inputenc}
\usepackage{amsmath}
\usepackage{indentfirst}
\usepackage{graphicx}
\usepackage{multicol,lipsum}

\begin{document}
%\maketitle

\begin{titlepage}
	\begin{center}
	
	%\begin{figure}[!ht]
	%\centering
	%\includegraphics[width=2cm]{c:/ufba.jpg}
	%\end{figure}

		\Huge{Instituto Federal do Piaui}\\
		\large{Departamento}\\ 
		\large{Programa}\\ 
		\vspace{15pt}
        \vspace{95pt}
        \textbf{\LARGE{Relatório Técnico de Software }}\\
		%\title{{\large{Título}}}
		\vspace{3,5cm}
	\end{center}
	
	\begin{flushleft}
		\begin{tabbing}
			Aluno: Francisco Edimar Furtado Melo Filho\\
			Professor orientador: Dr. Ricardo Ramos\\
			%Professor co-orientador: \\
	\end{tabbing}
 \end{flushleft}
	\vspace{1cm}
	
	\begin{center}
		\vspace{\fill}
			 Outubro\\
		 2019
			\end{center}
\end{titlepage}
%%%%%%%%%%%%%%%%%%%%%%%%%%%%%%%%%%%%%%%%%%%%%%%%%%%%%%%%%%%

% % % % % % % % %FOLHA DE ROSTO % % % % % % % % % %

\begin{titlepage}
	\begin{center}
	
	%\begin{figure}[!ht]
	%\centering
	%\includegraphics[width=2cm]{c:/ufba.jpg}
	%\end{figure}

		\Huge{Instituto Federal do Piaui}\\
		\large{Departamento}\\ 
		\large{Programa}\\ 
\vspace{15pt}
        
        \vspace{85pt}
        
		\textbf{\LARGE{Relatório Técnico de Software}}
		\title{\large{Título}}
	%	\large{Modelo\\
     %   		Validação do modelo clássico}
			
	\end{center}
\vspace{1,5cm}
	
	\begin{flushright}

   \begin{list}{}{
      \setlength{\leftmargin}{4.5cm}
      \setlength{\rightmargin}{0cm}
      \setlength{\labelwidth}{0pt}
      \setlength{\labelsep}{\leftmargin}}

      \item Relatório Técnico de Desenvolvimento de Software apresentado para a conclusão do Curso Tecnólogo em Analise e Desenvolvimento de Sistemas deste instituto.

      \begin{list}{}{
      \setlength{\leftmargin}{0cm}
      \setlength{\rightmargin}{0cm}
      \setlength{\labelwidth}{0pt}
      \setlength{\labelsep}{\leftmargin}}

			\item Aluno: Francisco Edimar Furtado Melo Filho
            \item Professor orientador: Dr. Ricardo Martins Ramos\

      \end{list}
   \end{list}
\end{flushright}
\vspace{1cm}
\begin{center}
		\vspace{\fill}
		 Outubro\\
		 2019
			\end{center}
\end{titlepage}
\newpage
% % % % % % % % % % % % % % % % % % % % % % % % % %
\newpage
\tableofcontents
\thispagestyle{empty}

\newpage
\pagenumbering{arabic}
% % % % % % % % % % % % % % % % % % % % % % % % % % %
\section{Resumo}
\newpage
\section{Apresentação}
\newpage

\section{Descrição de atividades}

\newpage
\section{Análise dos Resultados}
\newpage
\section{Trabalhos Futuros}
\newpage

\addcontentsline{toc}{section}{Bibliografia}
\section*{Bibliografia}
\footnotesize{

\noindent AGUIRRE, L. A. Introdução à Identificação de Sistemas, Técnicas Lineares e Não lineares Aplicadas a Sistemas Reais. Belo Horizonte, Brasil, EDUFMG. 2004.\\

}
\newpage
\addcontentsline{toc}{section}{Anexo}
\section*{Anexo}
\end{document}



